\chapter{�bersicht und Einf�hrung}

\begin{beschreibung}
In diesem Kapitel wollen wir die grundlegenden Begriffe f�r die stochastischen Prozesse definieren, eine kurze Einf�hrung in die Thematik geben und einfache Eigenschaften herleiten.
\end{beschreibung}

\section{Stochastische Prozesse}

Im ersten Schritt wollen wir uns den namensgebenden Begriff anschauen und festlegen, was wir unter einem stochastischen Prozess verstehen:

\begin{definition}[Stochastischer Prozess]\label{Nummer1.1.1}
Sei $(\Omega, \sA, P)$ ein Wahrscheinlichkeitsraum, $(\sX, \sB)$ ein Messraum und $T \neq \emptyset$. Dann hei�t eine Familie $X = (X_t)_{t \in T}$ mit f�r alle $t \in T$ messbaren Funktionen $X_t\colon \Omega \to \sX$ ein \deftxt{stochastischer Prozess}\index{Stochastischer Prozess}.
\end{definition}

Da der Begriff die Grundlage f�r die ganze Thematik darstellt, werden wir desweiteren einige Eigenschaften kennenlernen, um verschiedene Arten von stochastischen Prozessen zu unterscheiden.

\begin{klassifikation}\label{Nummer1.1.2}
Wir legen folgende klassifizierenden Eigenschaften f�r stochastische Prozesse fest:
\begin{itemize}
	\item Ist $\sX = \R$, so sprechen wir von \deftxt{reellwertigen}\index{Stochastischer Prozess!reellwertiger} stochastischen Prozessen.
	\item Ein stochastischer Prozess mit endlichem/abz�hlbarem \deftxt{Zustandsraum}\index{Zustandsraum} $\sX$ hei�t selbst \deftxt{endlich}\index{Stochastischer Prozess!endlicher}/\deftxt{abz�hlbar}\index{Stochastischer Prozess!abz�hlbarer}.
	\item Ist $\sX = \R^d$, so sprechen wir von einem \deftxt{Punktprozess}\index{Punktprozess}.
	\item Ist $T \in \{\N, \N_0, \Z\}$, so nennen wir den Prozess einen \deftxt{zeitdiskreten}\index{Stochastischer Prozess!zeitdiskreter} stochastischen Prozess.
	\item Falls $T \subset \R$ ein Intervall ist, so sprechen wir von einem \deftxt{zeitkontinuierlichen}\index{Stochastischer Prozess!zeitkontinuierlicher} stochastischen Prozess.
\end{itemize}
\end{klassifikation}

Wir werden uns im Wesentlichen jedoch auf zeitdiskrete/kontinuierliche stochastische Prozesse mit Zustandsraum $\R$ oder h�chstens abz�hlbaren Mengen beschr�nken, es gibt in der Theorie jedoch noch viel mehr F�lle wie z.\,B. Mengen von Funktionen als Zustandsraum.

\begin{beispiel*}[Bereits bekannte stochastische Prozesse]
Stochastische Prozesse sind zu diesem Zeitpunkt keine v�llig neuartigen Objekte, wir kennen bereits folgende Vertreter:
\begin{itemize}
	\item $(X_n)_{n \in \N}$ mit i.\,i.\,d. Zufallsvariablen $X_n$.
	\item $(\overline{X_n})_{n \in \N}$ f�r eine Folge von i.\,i.\,d. Zufallsvariablen $(X_n)_{n \in \N}$ mit $\overline{X_n} = \frac{1}{n}\sum_{i=1}^n X_i$.
	\item $(X_n^*)_{n \in \N}$ f�r i.\,i.\,d. Zufallsvariablen $X_n$ mit $X_n \in \sL_2$ und $\sigma^2 := \Var X_1 > 0$, sowie
	\begin{align*}
	X_n^* &:= \frac{1}{\sqrt{n\sigma^2}} \sum_{i=1}^n (X_i - \E X_i)\text{.}
	\end{align*}
\end{itemize}
Seien $(X_i)_{i \in \N}$ i.\,i.\,d. mit $X_1 \in \sL_2$ und $\Var X_1 > 0$, dann ist $(\overline{X_n})_{n \in \N}$ weder unabh�ngig noch identisch verteilt und $(X_n^*)_{n \in \N}$ nicht unabh�ngig, aber eventuell identisch verteilt.
\end{beispiel*}

\begin{definition}[Pfad/Trajektorie]\label{Nummer1.1.3}
Sei $X = (X_t)_{t \in T}$ ein stochastischer Prozess, dann hei�t f�r $\omega \in \Omega$ die Abbildung
\begin{align*}
X(\omega)\colon T &\to \sX\text{,}\\
t &\mapsto X_t(\omega)
\end{align*}
\deftxt{Pfad}\index{Pfad} oder \deftxt{Trajektorie}\index{Trajektorie} von $X$ bez�glich $\omega$.
\end{definition}

Stochastische Prozesse erzeugen also zuf�llige Abbildungen von $T$ in den Zustandsraum $\sX$. Abh�ngig von der Klasse des Prozesses (zeitdiskret, zeitkontinuerlich, \ldots) ist die erzeugte Funktion mitunter eine Folge oder eine "`normale"' reellwertige Funktion.

\begin{lemma}\label{Nummer1.1.4}
Wir betrachten die Menge $\sX^T := \bigtimes_{t \in T} \sX$ der Abbildungen $T \to \sX$, ausgestattet mit der Produkt-$\sigma$-Algebra $\sB^T := \bigotimes_{t \in T} \sB$. Ferner sei $X = (X_t)_{t \in T}$ ein $\sX$-wertiger stochastischer Prozess �ber $(\Omega, \sA, P)$. Dann ist
\begin{align*}
X\colon \Omega &\to \sX^T\text{,}\\
\omega &\mapsto X(\omega) = (t \mapsto X_t(\omega))
\end{align*}
eine messbare Abbildung, das hei�t $X$ ist eine $\sX^T$-wertige Zufallsvariable.
\end{lemma}

\begin{beweis}
Die Aussage des Lemma folgt unmittelbar aus \cite[Lemma I.9.16]{WT}.
\end{beweis}

Wir haben nun zwar einen neuen Begriff, allerdings noch kein Ziel, das wir anstreben. Daher wollen wir nun auf einige typische Fragestellungen eingehen. H�ufig wird untersucht, ob die Trajektorien beschr�nkt sind, wie ihr Wachstumsverhalten aussieht, ob sie f�r $t \to \infty$ konvergieren, ob sie stetig sind, wann ihre Austrittszeiten sind (d.\,h. wann verl�sst der Pfad ein gewisses Intervall) et cetera.

Stochastische Prozesse finden in vielen Gebieten breite Anwendung. Beispiele f�r typische Anwendungsfelder f�r stochastische Prozesse sind:
\begin{enumerate}
	\item \emph{Finanzmathematik} -- Bewertung von Finanzprodukten, Risikomangement, Investitionsstrategien
	\item \emph{Physik} -- Diffusionssysteme, stochastische Thermodynamik, Quantenphysik
	\item \emph{Biologie} -- Populationsmodelle, "`Genomics"'
	\item \emph{Ingenieurswissenschaften} -- Warteschlangenprobleme, Steuerungsprobleme
\end{enumerate}