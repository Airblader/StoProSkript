\section{Einfache Beispiele}

\begin{beispiel}[Unabh�ngig und identisch verteilt]\label{Nummer1.4.1}
Sei $X = (X_n)_{n \geq 1}$ i.\,i.\,d. Dann gilt:
\begin{enumerate}
	\item\label{Nummer141A1} Die Randverteilungen sind gerade die Produktma�e (siehe auch "`Kanonisches Modell"' in \cite[Satz II.2.5]{WT}).
	\item\label{Nummer141A2} $X$ ist station�r.
	\item\label{Nummer141A3} $X$ hat station�re Zuw�chse.
	\item\label{Nummer141A4} $X$ hat im Allgemeinen keine unabh�ngigen Zuw�chse.
\end{enumerate}
Die Aussagen \ref{Nummer141A2} und \ref{Nummer141A3} folgen aus \ref{Nummer141A1}, \ref{Nummer141A4} wird dem Leser zum Beweis �berlassen.
\end{beispiel}

\begin{beispiel}[Irrfahrt]\label{Nummer1.4.2}
Es sei $(Y_n)_{n \geq 0}$ ein i.\,i.\,d. stochastischer Prozess mit $Y_n \sim \Binom\left(1, \frac12\right)$. Dann hei�t $X = (X_n)_{n \geq 0}$ mit $X_n := \sum_{i=0}^n Y_i$ f�r $n \geq 0$ \deftxt{symmetrische Irrfahrt}\index{Irrfahrt!symmetrische}. Ist $Y_n \sim \Binom(1, p)$ f�r $p \in [0,1]$, so sprechen wir von einer \deftxt{asymmetrischen}\index{Irrfahrt!asymmetrische} Irrfahrt. Der Prozess $X$ hat station�re und unabh�ngige Zuw�chse, ist jedoch nicht station�r.

Zun�chst gilt f�r $m < n$, dass
\begin{align*}
X_n - X_m = \sum_{i=m+1}^n Y_i \tag{*}\label{Nummer142E1}
\end{align*} 
ist. Da die $Y_i$ unabh�ngig sind, folgt die Unabh�ngigkeit der Zuw�chse aus \cite[Satz II.2.7, Satz II.2.9]{WT} und der Tatsache, dass kein Index $i$ doppelt vorkommt. Die Stationarit�t der Zuw�chse folgt ebenfalls aus \eqref{Nummer142E1}. Da $X_n \sim \Binom(n, p)$ gilt sind die eindimensionalen Randverteilungen nicht gleich, d.\,h. $P_n \neq P_{n+1}$.\\
In der Regel wird die symmetrische Irrfahrt jedoch anders definiert, da man in beide Richtungen gehen k�nnen m�chte. Man setzt hierf�r $Z_n := \sum X_n - 1$. Dies entspricht der Summe von Zufallsvariablen mit Werten in $\{\pm 1\}$.

Die Irrfahrt wird u.\,a. zur Untersuchung von einfachen Wettspielen verwendet. Beispielsweise gebe es zwei Spieler und eine M�nze, die wiederholt geworfen wird. Falls sie Kopf zeigt, so verliert Spieler 1 und zahlt seinem Kontrahenten einen Euro, entsprechend umgekehrt f�r den Fall "`Zahl"'. Man kann sich nun fragen, wie lange es dauert, bis einer der Spieler pleite ist (ein gewisses Startkapital sei gegeben) oder wie gro� die Wahrscheinlichkeit daf�r ist, dass Spieler 1 verliert et cetera. \qedhere
\begin{figure}[!htb]
\centering
\begin{tikzpicture}
\draw[->, semithick] (0, -0.5) -- (0, 4);
\draw[->, semithick] (-0.5, 0) -- (7.5, 0);
\draw[circle, fill=blue] (0, 3) circle (2pt) node (AK) {};
\node[rotate=90] (Sp1) at (-0.5, 3) {\small Anfangskapital};
\draw[blue, semithick] (0, 3) -- (1, 2) -- (2, 3) -- (3, 2) -- (4, 1) -- (5, 2) -- (6, 1) -- (7, 0); 
\draw[circle, fill=blue] (7,0) circle (2pt) node (Pleite2) {};
\node (Pleite1) at (8, 1.5) {\small Pleite};
\draw[->] (Pleite1) to[out=-90, in=80] (Pleite2);
\end{tikzpicture}
\caption[Darstellung einer Irrfahrt]{Darstellung einer Wettspiel-Irrfahrt f�r einen Spieler.}\label{irrfahrt}
\end{figure}
\end{beispiel}

\begin{beispiel}[Gleitendes Mittel]\label{Nummer1.4.3}
Es sei $(X_n)_{n \in \Z}$ ein $\R$-wertiger stochastischer Prozess, $k \in \N$ und $c_0, \ldots, c_k \in [0,1]$ mit $\sum_{i=0}^k c_i = 1$. Wir definieren nun $Y = (Y_n)_{n \in \Z}$ verm�ge $Y_n := \sum_{i=0}^k c_i X_{n-i}$. Dann hei�t $Y$ \deftxt{gleitendes Mittel}\index{gleitendes Mittel} von $X$ bez�glich der Gewichtung $c_0, \ldots, c_k$. Gleitende Mittel gl�tten den Prozess $X$ gewisserma�en und werden unter anderem bei der Zeitreihenanalyse eingesetzt (z.\,B. Aktienkurse). Der Prozess $Y$ ist station�r. Gilt zudem, dass $(X_n)_{n \in \Z}$ i.\,i.\,d. mit $X_i \in \sL_2$ ist, so ist $\Var Y_n \leq \Var X_n$, wobei die echte Ungleichung in der Mehrzahl der F�lle gilt. Der Beweis f�r diese Eigenschaften wird dem Leser �berlassen.
\end{beispiel}

\begin{beispiel}[Untypisches Beispiel]\label{Nummer1.4.4}
Es sei $P = \bigotimes_{i=1}^\infty \lambda_{[0,1]}$ auf $[0,1]^\N$, ausgestattet mit der Produkt-$\sigma$-Algebra. Wir definieren f�r $\omega \in [0,1]$, also eine Folge $\omega = (\omega_i)_{i \geq 1}$, und $t \in [0, \infty)$ die Zufallsvariablen
\begin{align*}
X_t(\omega) &:= \sum_{i=1}^\infty \frac{\omega_i}{i!}t^i\text{.}
\end{align*}
Die Reihe konvergiert absolut und gleichm��ig auf allen Kompakta. Ferner bildet $X = (X_t)_{t \geq 0}$ einen stochastischen Prozess und jede Trajektorie $X(\omega) = (t \mapsto X_t(\omega))$ ist eine analytische Funktion. Ist also $X_t(\omega)$ f�r alle $t \in (t_1, t_2)$ bekannt, so ist bereits die gesamte Trajektorie bekannt. 
\end{beispiel}

Solche Prozesse wie in Beispiel \ref{Nummer1.4.4} interessieren uns hier allerdings nicht, weshalb wir sie nicht mehr betrachten werden. Stattdessen haben die Prozesse, die wir betrachten werden, eine wesentlich lockerere Beziehung zwischen Vergangenheit und Zukunft. Daf�r m�ssen wir jedoch erst mehr Theorie entwickeln.