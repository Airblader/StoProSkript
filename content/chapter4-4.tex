\section{Inhomogene Poissonprozesse}

Bisher war die Rate $\lambda$ konstant, also von der Zeit $t$ unabh�ngig. Wir wollen nun untersuchen, was passiert, wenn sie sich mit der Zeit �ndert.

\begin{definition}[Inhomogener Poissonprozess]\label{Nummer4.4.1}
Sei $(N_t)_{t \geq 0}$ ein rechtsstetiger, $\N_0$-wertiger stochastischer Prozess und $\lambda\colon [0, \infty) \to (0, \infty)$ Lebesgue-integrierbar auf allen kompakten Intervallen. Dann hei�t $(N_t)_{t \geq 0}$ ein \deftxt{nicht-homogener} oder \deftxt{inhomogener Poissonprozess}\index{Poissonprozess!inhomogener} genau dann, wenn folgende Eigenschaften gelten:
\begin{enumerate}
	\item\label{N441A1} Es gilt $N_0 = 0$.
	\item\label{N441A2} Der Prozess $(N_t)$ hat unabh�ngige Zuw�chse.
	\item\label{N441A3} F�r $s, t \geq 0$ ist der Zuwachs $N_{s+t} - N_s$ poissonverteilt mit der Rate $\displaystyle\int_s^{s+t} \hspace{-1em}\lambda(u)~\dd u$.
\end{enumerate}
\end{definition}

Wir haben also die Charakterisierung aus Satz \ref{Nummer4.2.1} verwendet, um von den homogenen auf die inhomogenen Poissonprozesse zu verallgemeinern. Jeder homogene Poissonprozess ist insbesondere ein inhomogener Poissonprozess mit konstanter Ratenfunktion $\lambda(t) := \lambda_0$. Den Nachweis zur Existenz und Eindeutigkeit echt inhomogener Poissonprozesse werden wir hier nicht f�hren und verweisen daher auf die Literatur.

\begin{definition}[Mittelwertfunktion]\label{Nummer4.4.2}
Ist $(N_t)$ ein inhomogener Poissonprozess mit Ratenfunktion $\lambda$, so hei�t $m\colon [0, \infty) \to (0, \infty)$ mit $t \mapsto \E N_t$ \deftxt{Mittelwertfunktion}\index{Mittelwertfunktion}.
\end{definition}

\begin{satz}[Eigenschaften der Mittelwertfunktion]\label{Nummer4.4.3}
Sei $(N_t)$ ein inhomogener Poissonprozess mit Ratenfunktion $\lambda$ und Mittelwertfunktion $m$. Dann gelten die folgenden Aussagen:
\begin{enumerate}
	\item\label{N443A1} Die Mittelwertfunktion $m$ ist eine Stammfunktion von $\lambda$, also gegeben durch
	\begin{align*}
	m(t) &= \int_0^t \lambda(u)~\dd u\text{.}
	\end{align*}
	\item\label{N443A2} Die Mittelwertfunktion $m$ ist streng monoton wachsend und stetig.
	\item\label{N443A3} Ist $m^*$ streng monoton wachsend und stetig, so existiert $\lambda^*$ mit $m^*(t) = \int_0^t \lambda^*(u)~\dd u$ f�r $t \geq 0$, also definiert $m^*$ einen inhomogenen Poissonprozess mit Ratenfunktion $\lambda^*$. Ferner ist $m^*$ Lebesgue-fast �berall differenzierbar und es gilt $(m^*)' = \lambda^*$ fast sicher, also ist $\lambda^*$ fast sicher eindeutig.
\end{enumerate}
\end{satz}

Im Fall homogener Poissonprozesse gilt also $m(t) = \E N_t = \lambda t$ und damit $m' = \lambda$.

\begin{beweis}
F�r \ref{N443A1} beachten wir, dass $N_t = N_t - N_0$ poissonverteilt mit Rate $\int_0^t \lambda(u)~\dd u$ ist. Dann gilt $m(t) = \E N_t = \int_0^t \lambda(u)~\dd u$, da der Erwartungswert $\alpha$-poissonverteilter Zufallsvariablen gerade $\alpha$ ist.

Die strenge Monotonie in \ref{N443A2} ist mit \ref{N443A1} klar, da $\lambda(t) > 0$ gilt. F�r die Stetigkeit betrachten wir f�r $0 \leq s \leq t \leq b$ das endliche Ma� $Q_b$ auf $\R$, welches die Lebesguedichte
\begin{align*}
h_b(u) &= \begin{cases}\lambda(u) & \text{falls } u \in [0,b]\\ 0 & \text{sonst}\end{cases}
\end{align*}
besitzt. Dann gilt
\begin{align*}
m(t) - m(s) &= \int_s^t \lambda(u)~\dd u = Q_b([s,t])
\end{align*}
und wir erhalten die Stetigkeit von $m$ aus der Stetigkeit von $Q_b$.

Ist $m^*$ streng monoton wachsend, so spricht man davon, dass $m^*$ von beschr�nkter Variation ist. Es folgt, dass $m^*$ fast �berall differenzierbar mit den Eigenschaften aus \ref{N443A3} ist. Wir wollen dies hier nicht ausf�hren und verweisen daher auf \cite{KESTELMAN} oder \cite{GRAVES}.
\end{beweis}

\begin{korollar}[Compound Process]\label{Nummer4.4.4}
Sei $(N_t)$ ein inhomogener Poissonprozess mit Mittelwertfunktion $m$. Dann gelten f�r den durch
\begin{align*}
C_t &:= \sum_{i=1}^{N_t} Y_i
\end{align*}
definierten Compound Process\index{Compound Process} von $N_t$ und einer von $(N_t)$ unabh�ngigen Folge mit i.\,i.\,d. Zufallsvariablen $Y_i$ die folgenden Eigenschaften:
\begin{enumerate}
	\item\label{N444A1} Ist $Y_1 \in \sL_1$, so folgt $C_t \in \sL_1$ und es gilt
	\begin{align*}
	\E C_t &= m(t) \E Y_1\text{.}
	\end{align*}
	\item\label{N444A2} Ist $Y_1 \in \sL_2$, so folgt $C_t \in \sL_2$ und es gilt
	\begin{align*}
	\Var C_t &= m(t) \E Y_1^2\text{.}
	\end{align*}
\end{enumerate}
Insbesondere sind $t \mapsto \E C_t$ und $t \mapsto \Var C_t$  streng monoton (oder konstant Null) und stetig.
\end{korollar}

\begin{beweis}
F�r den Erwartungswert erh�lt man mit Satz \ref{Nummer4.3.6}
\begin{align*}
\E C_t &= \E N_t \E Y_1 = m(t)\E Y_1\text{.}
\end{align*}

F�r die Varianz erh�lt man wegen $\E X = \alpha = \Var X$ f�r $X \sim \Pois(\alpha)$ wiederum mit Satz \ref{Nummer4.3.6}
\begin{align*}
\Var C_t &= \E N_t \Var Y_1 + \Var N_t (\E Y_1)^2 = m(t) \E Y_1^2\text{.} \qedhere
\end{align*}
\end{beweis}

\begin{korollar}[Summen]\label{Nummer4.4.5}
Sind $(N_t^{(1)})$ und $(N_t^{(2)})$ unabh�ngige, inhomogene Poissonprozesse mit Mittelwert- und Ratenfunktionen $m_1$ und $\lambda_1$ bzw. $m_2$ und $\lambda_2$, so ist $(N_t^{(1)} + N_t^{(2)})$ ein inhomogener Poissonprozess mit Mittelwertfunktion $m_1+m_2$ und Ratenfunktion $\lambda_1+\lambda_2$.
\end{korollar}

\begin{beweis}
F�r den Beweis, dass die Summe einen inhomogenen Poissonprozess definiert, m�ssen wir die drei Eigenschaften aus Definition \ref{Nummer4.4.1} nachweisen. Die erste Eigenschaft ist jedoch klar und die zweite Eigenschaft k�nnen wir wie in Satz \ref{Nummer4.3.1} zeigen. Die dritte und letzte Eigenschaft ist eine �hnliche Rechnung wie im Beweis von Satz \ref{Nummer4.3.1} und wird hier nicht bewiesen.
\end{beweis}

\begin{beispiel*}[Fortsetzung von Beispiel \ref{Nummer4.3.4} -- Teil 2]
Wieder betrachten wir eine Bank, der erste Eingang sei nun jedoch gegen�ber einer Bushaltestelle, der zweite Eingang m�nde in eine Fu�g�ngerzone und der dritte Eingang ist ein Zugang zu einem B�rohochhaus, dessen Angestellte zwischen 12 Uhr und 13 Uhr Mittagspause haben. Die Rechnungen verlaufen dann analog zu denen bei homogenen Poissonprozessen.
\end{beispiel*}