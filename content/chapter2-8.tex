\section{Optional Sampling}

Das Ziel dieses Abschnittes ist es, Optional Sampling f�r beliebige Stoppzeiten zu betreiben. Dies wird jedoch nicht f�r allgemeine, sondern f�r gleichm��ig integrierbare Martingale gelten.

Sei dazu $(X_n)_{n \geq 0}$ ein gleichm��ig integrierbares Martingal. Dann existiert $X_\infty \in \sL_1$ mit $X_n \to X_\infty$ $P$-fast sicher und in $\sL_1$. Ferner ist $X_n = \E(X_\infty \mid \sF_n)$. Betrachte nun eine beliebige Stoppzeit $\tau\colon \Omega \to \overline{\N_0}$. F�r $\omega \in \Omega$ mit $\tau(\omega) = \infty$ setzen wir $X_\tau(\omega) := X_\infty(\omega)$, ansonsten wie gewohnt als $X_\tau(\omega) := X_{\tau(\omega)}(\omega)$. Bis jetzt haben wir in Satz \ref{Nummer2.4.3} gesehen, dass wenn $\sigma \leq \tau$ Stoppzeiten sind und $\tau$ beschr�nkt ist, folgt, dass $\E(X_\tau \mid \sF_\sigma) = X_\sigma$ ist. Auf die Voraussetzung der Beschr�nktheit wollen wir nun verzichten:

\begin{satz}[Optional Sampling Theorem II]\label{Nummer2.8.1}\index{Optional Sampling Theorem}
Sei $(X_n)_{n \geq 0}$ ein gleichm��ig integrierbares Martingal und $\sigma \leq \tau$ Stoppzeiten. Dann gilt $X_\sigma, X_\tau \in \sL_1$ und
\begin{align*}
\E(X_\tau \mid \sF_\sigma) = X_\sigma\text{.}
\end{align*}
\end{satz}

\begin{beweis}
Zuerst wollen wir $X_\tau \in \sL_1$ zeigen. Da $(X_n)_{n \geq 0}$ gleichm��ig integrierbar ist, folgt mit Satz \ref{Nummer2.7.2}, dass $X_n = \E(X_\infty \mid \sF_n)$ f�r alle $n \geq 0$ gilt. Daraus folgt $|X_n| \leq \E(|X_\infty| \mid \sF_n)$. F�r $n \geq 0$ erhalten wir so
\begin{align*}
\E |X_n| \ind_{\{\tau = n\}} &\leq \E \E(|X_\infty| \mid \sF_n)\ind_{\{\tau = n\}} = \E |X_\infty| \ind_{\{\tau = n\}}\text{.} \tag{*}
\end{align*}
Ferner gilt
\begin{align*}
|X_\tau|\ind_{\{\tau \leq n\}} &\nearrow |X_\tau| \ind_{\{\tau < \infty\}} \tag{**}
\shortintertext{und}
|X_\tau|\ind_{\{\tau \leq n\}} &= \left|\sum_{k=0}^n X_\tau \ind_{\{\tau = k\}}\right| \leq \sum_{k=0}^n |X_k| \ind_{\{\tau = k\}}\text{.} \tag{***}
\end{align*}
Mit dem Satz von Beppo Levi\footnote{Dieser findet sich im Anhang als Satz \ref{appendix:beppolevi1}.} folgt nun
\begin{align*}
\E |X_\tau| \ind_{\{\tau < \infty\}} &\stackrel{\text{(**)}}{=} \lim_{n \to \infty} \E |X_\tau| \ind_{\{\tau \leq n\}}\\
\quad &\stackrel{\text{(***)}}{\leq} \sum_{k=0}^\infty \E |X_\tau| \ind_{\{\tau = k\}}\\
\quad &\stackrel{\text{(*)}}{\leq} \sum_{k=0}^\infty \E |X_\infty| \ind_{\{\tau = k\}}\\
\quad &= \E |X_\infty| \ind_{\{\tau < \infty\}}\\
\quad &\leq \E |X_\infty| < \infty\text{.}
\end{align*}
Au�erdem ist $\E |X_\tau| \ind_{\{\tau = \infty\}} \leq \E |X_\infty|$ und insgesamt erhalten wir $\E |X_\tau| \leq 2 \E|X_\infty| < \infty$, also gilt $X_\tau \in \sL_1$. Analog gilt dies nat�rlich auch f�r $X_\sigma \in \sL_1$. Im zweiten Schritt zeigen wir nun $\E(X_\tau \mid \sF_\sigma) = X_\tau$. Dazu verwenden wir Satz \ref{Nummer2.3.8}, der zeigte, dass $X_\tau$ auch $\sF_\sigma$-messbar ist. Der Satz zeigte dies nur f�r endliche Stoppzeiten, der Beweis ist nahezu wortw�rtlich aber auch auf unsere Situation hier �bertragbar. Wir wollen nun zun�chst $\E(X_\infty \ind_A) = \E(X_\tau\ind_A)$ f�r alle $A \in \sF_\tau$ zeigen. Sei dazu $n \geq 0$, dann sind $n \geq \tau \wedge n$ endliche Stoppzeiten und mit Satz \ref{Nummer2.4.3} erhalten wir $\E(X_n \mid \sF_{\tau \wedge n}) = X_{\tau \wedge n}$. Daraus folgt
\begin{align*}
\E(X_\infty \mid \sF_{\tau \wedge n}) &= \E(\E(X_\infty \mid \sF_n) \mid \sF_{\tau \wedge n}) = X_{\tau \wedge n}\text{.} \tag{$\heartsuit$}
\end{align*}
Ferner gilt f�r $A \in \sF_\tau$ nun $A \cap \{\tau \leq n\} = A \cap \{\tau \leq \tau \wedge n\} \in \sF_{\tau \wedge n}$ und es folgt
\begin{align*}
\E X_\infty \ind_{A \cap \{\tau \leq n\}} &= \E(\E X_\infty \ind_{A \cap \{\tau \leq n\}} \mid \sF_{\tau \wedge n})\\
\quad &= \E(\E(X_\infty \mid \sF_{\tau \wedge n}) \ind_{A \cap \{\tau \leq n\}})\\
\quad &\stackrel{\heartsuit}{=} \E X_{\tau \wedge n} \ind_{A \cap \{\tau \leq n\}}\text{.} \tag{$\heartsuit\heartsuit$}
\end{align*}
Ohne Einschr�nkung sei $X_\infty \geq 0$, sonst zerlegen wir wie �blich $X_\infty = X_\infty^+ - X_\infty^-$ und f�hren das folgende Argument zweimal durch. Nun folgt $X_n = \E(X_\infty \mid \sF_n) \geq 0$ und $X_\tau \geq 0$. Mit dem Satz von Beppo Levi\footnote{Dieser findet sich im Anhang als Satz \ref{appendix:beppolevi1}.} und dem Satz von Lebesgue\footnote{Dieser findet sich im Anhang als Satz \ref{appendix:lebesgue}.} folgt wegen
\begin{align*}
X_\infty \ind_{A \cap \{\tau \leq n\}} &\longrightarrow X_\infty \ind_{A \cap \{\tau < \infty\}}
\shortintertext{und}
X_{\tau \wedge n} \ind_{A \cap \{\tau \leq n\}} &\longrightarrow X_\tau \ind_{A \cap \{\tau < \infty\}}
\end{align*}
daher mit Hilfe von ($\heartsuit\heartsuit$) die folgende Identit�t:
\begin{align*}
\underbrace{\E X_\infty \ind_{A \cap \{\tau \leq n\}}}_{\to \E X_\infty\ind_{A \cap \{\tau < \infty\}}} &= \underbrace{\E X_{\tau \wedge n} \ind_{A \cap \{\tau \leq n\}}}_{\to \E X_\tau \ind_{A \cap \{\tau < \infty\}}}\text{.}
\end{align*}
Nach Definition gilt au�erdem $X_\tau\ind_{\{\tau = \infty\}} = X_\infty = X_\infty \ind_{\{\tau = \infty\}}$, wir erhalten damit also nun $X_\tau \ind_{A \cap \{\tau = \infty\}} = X_\infty \ind_{A \cap \{\tau = \infty\}}$ und insgesamt $\E X_\infty \ind_A = \E X_\tau \ind_A$. Dann folgt $\E(X_\infty \mid \sF_\tau) = X_\tau$ und damit
\begin{align*}
\E(X_\tau \mid \sF_\sigma) &= \E(\E(X_\infty \mid \sF_\tau) \mid \sF_\sigma) = \E(X_\infty \mid \sF_\sigma) = X_\sigma\text{.} \qedhere
\end{align*}
\end{beweis}