\section{Grundlegende Eigenschaften}

In diesem Abschnitt wollen wir zwei Ziele erreichen. Zun�chst wollen wir einige M�glichkeiten kennenlernen, aus Martingalen neue Martingale zu konstruieren. Danach wollen wir zeigen, dass stochastische Prozesse im Wesentlichen aus Martingalen und vorhersagbaren stochastischen Prozessen bestehen.

\begin{satz}\label{Nummer2.2.1}
Es seien $X = (X_n)_{n \geq 0}$ und $Y = (Y_n)_{n \geq 0}$ reellwertige, $\sF$-adaptierte stochastische Prozesse, wobei $\sF$ eine Filtration ist. Dann gilt:
\begin{enumerate}
	\item\label{Nummer221A1} Sind $X$ und $Y$ Martingale und $\alpha, \beta \in \R$, so ist auch $\alpha X + \beta Y$ ein Martingal.
	\item\label{Nummer221A2} Sind $X$ und $Y$ beide Super- oder Sub-Martingale und $\alpha, \beta \geq 0$, so ist auch $\alpha X + \beta Y$ ein Super- bzw. Sub-Martingal.
	\item\label{Nummer221A3} Sind $X$ und $Y$ Super-Martingale, so ist $X \wedge Y := \min\{X, Y\}$ ebenfalls ein Super-Martingal. Sind $X$ und $Y$ Sub-Martingale, so ist analog $\max\{X, Y\}$ ein Sub-Martingal.
	\item\label{Nummer221A4} Ist $X$ ein Super-Martingal, $(T_n)_{n \geq 1} \subset \N$ mit $T_n \to \infty$ und $\E X_{T_n} \geq \E X_0$ f�r alle $n \geq 1$, so ist $X$ ein Martingal.
\end{enumerate}
\end{satz}

\begin{beweis}
Die ersten beiden Eigenschaften lassen sich durch simples Nachrechnen unter Verwendung der Linearit�t des bedingten Erwartungswertes beweisen. F�r \ref{Nummer221A3} setzen wir $Z_n := X_n \wedge Y_n$. Wegen $|Z_n| \leq |X_n| + |Y_n|$ folgt dann $Z_n \in \sL_1$ und die Tatsache, dass $Z_n$ adaptiert ist. Dann erhalten wir
\begin{align*}
\E(Z_n~|~\sF_{n-1}) &\leq \E(X_n~|~\sF_{n-1}) \leq \E(X_{n-1}~|~\sF_{n-1}) \leq \E(Y_{n-1}~|~\sF_{n-1}) \leq \E(Z_{n-1}~|~\sF_{n-1})\text{.}
\end{align*} 
Nun wollen wir \ref{Nummer221A4} beweisen. Dazu sei $m \in \N$, dann existiert ein $n \in \N$ mit $m < T_n$ und wir setzen $Y_i := \E(X_{T_n}~|~\sF_i)$ f�r $i < T_n$. Zuerst zeigen wir, dass $P$-fast sicher $X_i = Y_i$ f�r alle $i < T_n$ gilt. Dazu betrachte
\begin{align*}
Y_i &= \E(X_{T_n}~|~\sF_i) = \E(\E(X_{T_n}~|~\sF_{T_n - 1})~|~\sF_i)\\
		&\leq \E(X_{T_n-1}~|~\sF_i)\\
		&\phantom{\leq}\quad\vdots\\
		&\leq \E(X_i~|~\sF_i) = X_i\text{.}
\end{align*}
Mit dieser Rechnung gilt nun
\begin{align*}
\E X_0 &\leq \E X_{T_n} = \E(\E( X_{T_n}~|~\sF_i)) = \E Y_i\\
			 &\leq \E X_i = \E(\E(X_i~|~\sF_{i-1}))\\
			 &\leq \E X_{i-1}\\
			 &\phantom{\leq}\quad\vdots\\
			 &\leq \E X_0\text{.}
\end{align*}
Damit m�ssen all diese Ungleichungen also tats�chlich Gleichungen sein und wir erhalten insbesondere $\E X_i = \E Y_i$ f�r alle $i < T_n$. Fassen wir dies zusammen, so erhalten wir $X_i = Y_i$ $P$-fast sicher f�r alle $i < T_n$. Nun gilt
\begin{align*}
\E(X_m~|~\sF_{m-1}) &= \E(\underbrace{\E(X_{T_n}~|~\sF_m)}_{= Y_m}~|~\sF_{m-1}) = \E(X_{T_n}~|~\sF_{m-1}) = Y_{m-1} = X_{m-1}\text{.} \qedhere
\end{align*}
\end{beweis}

\begin{satz}\label{Nummer2.2.2}
Es sei $X = (X_n)_{n \geq 0}$ ein Martingal und $\phi\colon \R \to \R$ eine konvexe Abbildung. Dann gilt:
\begin{enumerate}
	\item\label{Nummer222A1} Falls f�r den positiven Anteil der Komposition $\E(\phi \circ X)^+ < \infty$ f�r alle $n \geq 0$ gilt, so folgt, dass $\phi \circ X$ ein Sub-Martingal ist.
	\item\label{Nummer222A2} Ist $X_n \in \sL_p$ f�r ein $p \in [1, \infty)$ und alle $n \geq 0$, so ist $|X|^p$ ein Sub-Martingal.
\end{enumerate}
\end{satz}

\begin{beweis}
Die Aussage \ref{Nummer222A2} folgt aus \ref{Nummer222A1} f�r $\phi(t) := |t|^p$. Beweisen wir also \ref{Nummer222A1}. Dazu sei $v(x) = ax+b$ affin linear f�r $x \in \R$ mit $v \leq \phi$ (vgl. Beweis von Satz \ref{Nummer1.5.5}). Dann gilt $\phi^- = \max\{0, -\phi\} \leq \max\{0, -v\} = v^-$ und wir erhalten
\begin{align*}
\E(\phi \circ X_n)^- &\leq \E(v \circ X)^- \leq |a| \E |X_n| + |b| < \infty\text{.}
\end{align*}
Dann ist $\phi \circ X_n \in \sL_1$. Nun folgt mit Satz \ref{Nummer1.5.5}
\begin{align*}
\E(\phi \circ X_n~|~\sF_{n-1}) &\geq \phi \E(X_n~|~\sF_{n-1}) = \phi \circ X_{n-1}\text{.} \qedhere
\end{align*}
\end{beweis}

\begin{lemma}\label{Nummer2.2.3}
Es sei $M = (M_n)_{n \geq 0}$ ein $\sF$-adaptiertes, vorhersagbares Martingal. Dann gilt $P$-fast sicher $M_n = M_0$ f�r alle $n \geq 0$.
\end{lemma}

\begin{beweis}
Es gen�gt, induktiv zu zeigen, dass $P$-fast sicher $M_n = M_{n-1}$ f�r alle $n \geq 1$ gilt. Dies gilt aber wegen $M_n = \E(M_n~|~\sF_{n-1}) = M_{n-1}$.
\end{beweis}

\begin{lemma}\label{Nummer2.2.4}
Es sei $M = (M_n)_{n \geq 0}$ ein Martingal, dann folgt $\E(M_n \cdot M_m) = \E M_m^2$ f�r alle $0 \leq m \leq n$.
\end{lemma}

\begin{beweis}
Wir k�nnen ohne Einschr�nkung $n > m$ annehmen, da der Fall $n = m$ trivial ist. Dann gilt mit iterierter Anwendung der Martingaleigenschaft
\begin{align*}
\E(M_n \cdot M_m) &= \E(\E(M_n \cdot M_m ~|~ \sF_m)) = \E(M_m(\underbrace{\E(M_n~|~\sF_m)}_{= M_m})) = \E(M_m \cdot M_m)\text{.} \qedhere
\end{align*}
\end{beweis}

\begin{satz}[Doob-Zerlegung]\index{Doob!-Zerlegung}\label{Nummer2.2.5}
Es sei $X = (X_n)_{n \geq 0}$ ein an $\sF = (\sF_n)_{n \geq 0}$ adaptierter stochastischer Prozess mit $X_n \in \sL_1(P)$ f�r alle $n \geq 0$. Dann existiert -- bis auf Ununterscheidbarkeit -- genau ein Martingal $M$ und ein $\sF$-vorhersagbarer stochastischer Prozess $A$ mit $A_0 = 0$, so dass $X = M + A$ gilt. Ferner ist $X$ ein Sub-Martingal genau dann, wenn $A$ monoton wachsend ist.
\end{satz}

Mit anderen Worten bestehen stochastische Prozesse also aus Martingalen und vorhersagbaren Prozessen.

\begin{beweis}
Wir zeigen zun�chst die Existenz. F�r $n \geq 0$ sei dazu
\begin{align*}
M_n &:= X_0 + \sum_{k=1}^n \left(X_k - \E(X_k~|~\sF_{k-1})\right)\text{,}
\shortintertext{sowie}
A_n &:= -\sum_{k=1}^n \left(X_{k-1} - \E(X_k~|~\sF_{k-1})\right)\text{.}
\end{align*}
Dann gilt, dass $A_n$ nach Konstruktion $\sF_{n-1}$-messbar ist, da es als Summe �ber $\sF_{k-1}$-messbare Terme entsteht und $\sF_{k-1} \subset \sF_{n-1}$ f�r $k \in \{1, \ldots, n\}$ gilt. Dann folgt, dass $A$ $\sF$-vorhersagbar ist; dass $A_0 = 0$ gilt ist klar. Ferner ist $M_n$ nach Konstruktion mit analoger Argumentation $\sF_n$-messbar und integrierbar, wir m�ssen also noch die Martingaleigenschaft zeigen. Dazu betrachten wir
\begin{align*}
\E(M_n - M_{n-1}~|~\sF_{n-1}) &= \E(X_n - \E(X_n~|~\sF_{n-1})~|~\sF_{n-1}) = \E(X_n~|~\sF_{n-1}) - \E(X_n~|~\sF_{n-1}) = 0\text{.}
\end{align*}
Nun bleibt noch zu zeigen, dass $M$ und $A$ tats�chlich $X$ zerlegen. Dies gilt wegen
\begin{align*}
M_n + A_n &= X_0 + \sum_{k=1}^n (X_k - \E(X_k~|~\sF_{k-1})) - \sum_{k=1}^n X_{k-1} + \sum_{k=1}^n \E(X_k~|~\sF_{k-1})\\
					&= X_n\text{.}
\end{align*}
Damit ist die Existenz gezeigt und wir zeigen die Eindeutigkeit. Dazu sei $X = M + A = M' + A'$, dann gilt $M - M' = A' - A$. Nach Satz \ref{Nummer2.2.1} ist $M - M'$ ein Martingal und $A' - A$ ist vorhersagbar, also auch $M - M'$. Mit Lemma \ref{Nummer2.2.3} folgt dann $M_n - M_n' = M_0 - M_0' = A_0 - A_0' = 0 - 0 = 0$, also folgt $P$-fast sicher $M = M'$.

Zum Schluss beweisen wir noch, dass $X$ ein Sub-Martingal ist genau dann, wenn $A$ wachsend ist. Dazu �berlegen wir uns zun�chst
\begin{align*}
\E(X_n~|~\sF_{n-1}) &= \E(M_n~|~\sF_{n-1}) + \E(A_n~|~\sF_{n-1}) = M_{n-1} + A_n\text{.}
\end{align*}
Also ist $X$ ein Sub-Martingal. Daraus folgt $M_{n-1} + A_n = \E(X_n~|~\sF_{n-1}) \geq X_{n-1} = M_{n-1} + A_{n-1}$ und wir erhalten $A_n \geq A_{n-1}$. Die andere Richtung folgt analog.
\end{beweis}

\begin{korollar}\label{Nummer2.2.6}
Es sei $X = (X_n)_{n \geq 0}$ ein quadrat-integrierbares $\sF$-Martingal, d.\,h. es gilt $X_n \in \sL_2$ f�r alle $n \geq 0$. Dann gilt:
\begin{enumerate}
	\item\label{Nummer226A1} $X^2$ ist ein Sub-Martingal.
	\item\label{Nummer226A2} Es gibt genau einen $\sF$-vorhersagbaren stochastischen Prozess $A$ mit $A_0 = 0$, so dass $X^2-A$ ein Martingal ist. Dieser Prozess $A$ hei�t \deftxt{quadratischer Variationsprozess}\index{Quadratischer Variationsprozess} von $X$ und wird mit $\langle X \rangle := A$ bezeichnet.
	\item\label{Nummer226A3} Der Prozess $\langle X \rangle$ ist monoton wachsend.
	\item\label{Nummer226A4} Es gilt $\displaystyle\langle X \rangle_n = \sum_{i=1}^n \E((X_i - X_{i-1})^2~|~\sF_{i-1})$ und $\displaystyle \E\langle X \rangle_n = \Var (X_n - X_0)$.
\end{enumerate}
\end{korollar}

\begin{beweis}
Aussage \ref{Nummer226A1} folgt unmittelbar aus Satz \ref{Nummer2.2.2}. Aussage \ref{Nummer226A2} folgt aus Satz \ref{Nummer2.2.5}, denn f�r $X^2$ mit $X^2 = M + A$ erhalten wir $X^2 - A = M$. Aussage \ref{Nummer226A3} folgt ebenfalls aus Satz \ref{Nummer2.2.5}. Wir zeigen also noch \ref{Nummer226A4}. Dazu betrachten wir
\begin{align*}
\sum_{i=1}^n \E((X_i - X_{i-1})^2~|~\sF_{i-1}) &= \sum_{i=1}^n \E(X_i^2 - 2X_iX_{i-1} + X_{i-1}^2~|~\sF_{i-1})\text{,}
\shortintertext{durch Auseinanderziehen und Anwenden der Definitionen der bedingten Erwartung und Martingalen erhalten wir dann}
\quad &= \sum_{i=1}^n \left(\E(X_i^2~|~\sF_{i-1}) - X_{i-1}^2\right) \stackrel{\text{\ref{Nummer2.2.5}}}{=} A_n\\
\quad &= \langle X \rangle_n\text{.}
\end{align*}
Nun wollen wir noch die Formel f�r den Erwartungswert beweisen. Es gilt
\begin{align*}
\E A_n &= \sum_{i=1}^n \E(\E(X_i - X_{i-1})^2~|~\sF_{i-1}) = \sum_{i=1}^n \E\left(X_i^2 - 2X_iX_{i-1} + X_{i-1}^2\right)\text{,}
\shortintertext{und wieder durch Auseinanderziehen und Lemma \ref{Nummer2.2.4} erhalten wir}
\quad &= \sum_{i=1}^n \left(\E X_i^2 - \E X_{i-1}^2\right) = \E X_n^2 - \E X_0^2 = \E X_n^2 - 2\E X_nX_0 + \E X_0^2 = \E (X_n - X_0)^2\\
\quad &= \Var (X_n - X_0)\text{,}
\end{align*}
da $\E X_n = \E X_0$ gilt, wie wir bereits gesehen haben.
\end{beweis}

\begin{beispiel}\label{Nummer2.2.7}
Es seien $(Y_i)_{i \geq 1}$ unabh�ngig mit $Y_i \in \sL_2(P)$ und $\E Y_i = 0$ f�r alle $i \geq 0$, sowie $X_n := \sum_{i=1}^n Y_i$ f�r $n \geq 0$. Dann gilt
\begin{align*}
\langle X \rangle_n &= \sum_{i=1}^n \E Y_i^2 = \sum_{i=1}^n \Var Y_i\text{.} \qedhere
\end{align*}
\end{beispiel}

\begin{beispiel}\label{Nummer2.2.8}
Es seien $(Y_i)_{i \geq 0}$ unabh�ngig mit $Y_i \in \sL_2(P)$ und $\E Y_i = 1$ f�r alle $i \geq 0$, sowie $X_n := \prod_{i=1}^n Y_i$ f�r $n \geq 0$. Dann gilt
\begin{align*}
\langle X \rangle_n &= \sum_{i=1}^n X_{i-1}^2 \Var Y_i\text{.} \qedhere
\end{align*}
\end{beispiel}