\section{Konvergenzs�tze f�r Martingale I}

In diesem Abschnitt wollen wir untersuchen, ob ein $X$ existiert, so dass $X_n \to X$ gilt, falls $X_n$ ein (Super-/Sub-)Martingal ist.

Wir wollen f�r das kommende Lemma \ref{Nummer2.5.1} einen Ansatz diskutieren. Dazu sei $(x_n) \subset \R$ eine Folge, die weder eigentlich noch uneigentlich konvergiert. Dann ist $\liminf x_n < \limsup x_n$ und es seien $a < b$ mit $\liminf x_n < a < b < \limsup x_n$. Das Intervall $[a,b]$ wird von der Folge dann unendlich oft �berquert. Um Konvergenz zu zeigen, gen�gt es daher, die Anzahl der �berquerungen abzusch�tzen. In unserem Fall haben wir es nat�rlich mit Prozessen und nicht mit einfachen Zahlenfolgen zu tun.

Sei nun $X$ ein $\sF$-adaptierter stochastischer Prozess und $a < b$. Dann definieren wir $\sigma_0 := 0$, $\tau_k = \inf\{n \geq \sigma_{k-1} : X_n \leq a\}$ und $\sigma_k = \inf\{n \geq \tau_k : X_n \geq b\}$. Der Nachweis, dass dies tats�chlich Stoppzeiten definiert, wird zur �bung �berlassen. Ferner gilt $\tau_k \leq \sigma_k \leq \tau_{k+1}$. Nun gibt $\tau_1$ an, wann das erste Mal $X_n \leq a$ gilt, w�hrend $\sigma_1$ den ersten Zeitpunkt angibt, zu welchem $X_n \geq b$ gilt, nachdem $X_n \leq a$ war. Dann gibt $\tau_2$ den zweiten Zeitpunkt mit $X_n \leq a$ an, nachdem $X_n \geq b$ eingetreten ist et cetera. In Abbildung \ref{uebquerstopp} wird dies veranschaulicht.

\begin{figure}[!htb]
\centering
\begin{tikzpicture}[domain=-5:5, scale=1]
\draw[semithick, ->] (-0.5,0) -- (14,0);
\draw[semithick, ->] (0,-0.5) -- (0,6);
\draw[semithick] (-0.1, 1.7) node[left] {$a$} -- (0.1, 1.7);
\draw[semithick] (-0.1, 4.3) node[left] {$b$} -- (0.1, 4.3);
\draw[dashed] (0.1, 1.7) -- (14, 1.7);
\draw[dashed] (0.1, 4.3) -- (14, 4.3);
\begin{scope}[shift={(7,0.5)}]
\draw[color=red]   plot file{tikz/uebquer.table};
\draw[dashed] (-5.53, 1.2) -- (-5.53, -0.6) node[below] {$\tau_1$};
\draw[dashed] (-2.68, 3.8) -- (-2.68, -0.6) node[below] {$\sigma_1$};
\draw[dashed] (0.31, 1.2) -- (0.31, -0.6) node[below] {$\tau_2$};
\draw[dashed] (4.6, 3.8) -- (4.6, -0.6) node[below] {$\sigma_2$};
\end{scope}
\end{tikzpicture}
\caption[Darstellung der �berquerungsstoppzeiten]{Darstellung der �berquerungsstoppzeiten.}\label{uebquerstopp}
\end{figure}

F�r $N \in \N_0$ betrachten wir ferner $U_{a,b}^N(\omega) := \sup\{k \in \N_0 : \sigma_k(\omega) \leq N\}$, was die Anzahl der \deftxt{Aufw�rts�berquerungen}\index{Aufw�rts�berquerung}, die wir auch \deftxt{Upcrossings}\index{Upcrossing} nennen, darstellt. Nach obigem Plan wollen wir nun $U_{a,b}^N$ absch�tzen.

\begin{lemma}\label{Nummer2.5.1}
Sei $X = (X_n)_{n \geq 0}$ ein Sub-Martingal, $N \in \N_0$ und $a < b$. Dann gilt
\begin{align*}
\E U_{a,b}^N &\leq \frac{\E(X_N - a)^+}{b-a}\text{.}
\end{align*}
\end{lemma}

\begin{beweis}
Wir betrachten $Y_n := (X_n - a)^+$, dann ist $(Y_n)_n$ ein Sub-Martingal nach Satz \ref{Nummer2.2.1} und es gilt $Y_n \geq 0$. Ohne Einschr�nkung k�nnen wir daher $X_n \geq 0$ und $a = 0$ annehmen. Nun setzen wir
\begin{align*}
H_n &:= \sum_{k=1}^\infty \ind_{\underbrace{\{\tau_k < n \leq \sigma_k\}}_{\in \sF_{n-1}}} \qquad \text{f�r } n \geq 1\text{,}
\end{align*}
dann ist $(H_n)_n$ ein vorhersagbares Martingal mit $H_n \geq 0$ f�r alle $n \geq 1$. Ferner gilt $H_n \leq 1$, denn f�r $\omega \in \{\tau_k < n \leq \sigma_k\}$ gilt wegen $\sigma_k \leq \tau_{k+1}$, dass $\tau_{k+1}(\omega) \geq n$ ist, also ist $\omega \notin \{\tau_{k+1} < n \leq \sigma_{k+1}\}$. Mit Lemma \ref{Nummer2.1.9} folgt nun, dass $(1-H).X$ ein Sub-Martingal ist und wir erhalten $\E((1-H).X)_N \geq \E((1-H).X)_0 = \E X_0$. Ferner gilt
\begin{align*}
((1-H).X)_N &= X_0 + \sum_{m=1}^N (1-H_m)(X_m - X_{m-1}) = X_N - \sum_{m=1}^N H_m(X_m - X_{m-1})\\
					  &= X_N + X_0 - (H.X)_N\text{.}
\end{align*}
Daraus folgt $\E(H.X)_N \leq \E X_N$. Weiter gilt nun
\begin{align*}
H_k &= \begin{cases}1 & \text{falls } k \in \{\tau_i + 1, \ldots, \sigma_i\} \text{ f�r ein } i \in \N\\ 0 & \text{sonst}\end{cases}\text{.}
\end{align*}
F�r $m \geq 1$ und $\sigma_m < \infty$ gilt dann unter Verwendung dieser Eigenschaft
\begin{align*}
(H.X)_{\sigma_m} &= X_0 + \sum_{k=1}^{\sigma_m} H_k(X_k - X_{k-1}) = X_0 + \sum_{i=1}^m \sum_{k = \tau_i+1}^{\sigma_i} (X_k - X_{k-1})\\
								 &= X_0 + \sum_{i=1}^m (X_{\sigma_i} - X_{\tau_i})\\
								 &\geq bm\text{,}
\end{align*}
wobei die letzte Absch�tzung wegen $X_{\sigma_i} \geq b$ und $X_{\tau_i} \leq a = 0$ gilt. Sei nun $N \in \N_0$, dann betrachten wir zwei F�lle. Im ersten Fall ist $N \in \{\sigma_m, \ldots, \tau_{m+1}\}$  f�r ein $m \in \N$. Dann gilt $U_{a,b} = m$ und damit folgt
\begin{align*}
(H.X)_N &= X_0 + \sum_{k=1}^N H_k(X_k - X_{k-1}) = X_0 + \sum_{k=1}^{\sigma_m} H_k(X_k - X_{k-1}) \geq bm = b U_{a,b}^N\text{.}
\end{align*}
Im zweiten Fall sei $N \in \{\tau_m+1, \ldots, \sigma_m\}$ f�r ein $m \in \N$. Dann ist $U_{a,b}^N = m-1$ und wir erhalten
\begin{align*}
(H.X)_N &= X_0 + \sum_{k=1}^N H_k(X_k - X_{k-1}) = X_0 + \sum_{k=1}^{\tau_m} H_k(X_k - X_{k-1}) + \sum_{k=\tau_m+1}^N H_k(X_k - X_{k-1})\\
			  &= X_0 + \sum_{k=1}^{\sigma_{m-1}} H_k(X_k - X_{k-1}) + X_N - X_{\tau_m} = (H.X)_{\sigma_{m-1}} + X_N - X_{\tau_m}\\
				&\geq (H.X)_{\sigma_{m-1}} = b(m-1) = bU_{a,b}^N\text{.}
\end{align*}
Nehmen wir beide F�lle zusammen, so erhalten wir also $(H.X)_N \geq bU_{a,b}^N$ und damit
\begin{align*}
b\E U_{a,b}^N &\leq \E (H.X)_N \leq \E X_N\text{.} \qedhere
\end{align*}
\end{beweis}

\begin{satz}[Martingalkonvergenz I]\label{Nummer2.5.2}\index{Martingalkonvergenzsatz}
Es sei $(X_n)_{n \geq 0}$ ein Sub-Martingal mit $\sup_{n \geq 0} \E X_n^+ < \infty$. Wir setzen $\sF_\infty := \sigma\left(\bigcup_{n = 0}^\infty \sF_n\right)$. Dann existiert eine $\sF_\infty$-messbare Zufallsvariable $X_\infty \in \sL_1$ mit $X_n \to X_\infty$ $P$-fast sicher.
\end{satz}

\begin{beweis}
Sei $a < b$. Dann gilt $\E (X_n - a)^+ \leq |a| + \E X_n^+ < c < \infty$ f�r ein $c > 0$ und alle $n \geq 0$. Mit Lemma \ref{Nummer2.5.1} folgt nun
\begin{align*}
\E U_{a,b}^n &\leq \frac{|a| + \E X_n^+}{b-a} < \frac{c}{b-a} < \infty\text{.}
\end{align*}
F�r $n \to \infty$ gilt ferner $U_{a,b}^n \nearrow U_{a,b} := \lim U_{a,b}^n$ $P$-fast sicher. Mit dem Satz von Beppo Levi\footnote{Dieser findet sich im Anhang als Satz \ref{appendix:beppolevi1}.} folgt dann
\begin{align*}
\E U_{a,b} &= \lim_{n \to \infty} \E U_{a,b}^n \leq \frac{c}{b-a} < \infty\text{.}
\end{align*}
Damit erhalten wir $P(U_{a,b} < \infty) = 1$. Sei nun $C^{a,b} := \{\liminf X_n < a\} \cap \{\limsup X_n > b\}$. Es ist klar, dass $X_n(\omega)$ f�r $\omega \in C^{a,b}$ nicht konvergiert. Ferner gilt offenbar $C^{a,b} \subset \{U_{a,b} = \infty\}$, was aber eine Nullmenge ist. F�r $C := \bigcup_{a < b \in \Q} C^{a,b}$ gilt also $P(C) = 0$. Nach Konstruktion ist jedoch $X_n$ auf $\Omega\setminus C$ konvergent. Daher existiert $X_\infty$, so dass $P$-fast sicher $X_n \to X_\infty$ gilt. 

Da $X_n$ wegen $\sF_n \subset \sF_\infty$ auch $\sF_\infty$-messbar ist, folgt, dass $X_\infty$ auch $\sF_\infty$-messbar ist. Da $X$ ein Sub-Martingal ist, folgt zudem $\E X_0 \leq \E X_n = \E X_n^+ - \E X_n^-$. Dann ist $\E |X_n| = \E X_n^+ + \E X_n^- \leq \E X_n^+ + \E X_n^+ - \E X_0 < c'$ f�r ein $c' < \infty$. Mit dem Lemma von Fatou\footnote{Dieses findet sich im Anhang als Lemma \ref{appendix:fatou}.} gilt dann $\E |X|_\infty \leq \liminf \E |X_n| \leq c' < \infty$. 
\end{beweis}

\begin{korollar}\label{Nummer2.5.3}
Ist $X$ ein nicht-negatives Super-Martingal, so existiert eine $\sF_\infty$-messbare Zufallsvariable $0 \leq X_\infty \in \sL_1$ mit $\E X_\infty \leq \E X_0$ und $X_n \to X_\infty$ $P$-fast sicher.
\end{korollar}

\begin{beweis}
Offenbar ist $-X \leq 0$ ein Sub-Martingal. Dann ist $\E (-X_n)^+ \leq 0$ f�r alle $n \geq 1$. Mit Satz \ref{Nummer2.5.2} existiert dann ein $0 \geq -X_\infty \in \sL_1$, das $\sF_\infty$-messbar ist, mit $-X_n \to -X_\infty$ $P$-fast sicher. Mit dem Lemma von Fatou\footnote{Dieses findet sich im Anhang als Lemma \ref{appendix:fatou}.} gilt zudem $\E X_\infty \leq \liminf \E X_n \leq \E X_0$.
\end{beweis}

Aus der Wahrscheinlichkeitstheorie kennen wir verschiedene Konvergenzbegriffe. Da wir nun immer integrierbare Zufallsvariablen vorliegen haben, wollen wir untersuchen, wann $X_n \to X_\infty$ in $\sL_1$ gilt. Dazu m�ssen wir im n�chsten Abschnitt jedoch einige Vorbereitungen treffen.