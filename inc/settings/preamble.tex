% PREAMBLE
% =========
					
\usepackage[ngerman]{babel}
\usepackage[latin1]{inputenc}					% input encoding
\usepackage[T1]{fontenc}							% use T1 fonts for font encoding
\usepackage{amsfonts}									% math font
\usepackage{newcent}									% use New Century Schoolbook
\usepackage{fouriernc}								% math font for newcent
\usepackage{dsfont}										% math font for number spaces etc.
\usepackage{helvet}										% sans serif font
\usepackage{inconsolata}							% font for typewriter style

% general
\usepackage{makeidx}
\makeindex

\usepackage{ifthen}
\usepackage{letltxmacro}							% better support for redefining macros with optional arguments
\usepackage{booktabs}									% nicer tables
\usepackage{wrapfig}									% text around figures
\usepackage{parskip}									% vertical space instead of indentation
\usepackage{microtype}								% improved justification
\usepackage{needspace}								% reserve space
\usepackage{framed}										% shaded environments
\setlength{\FrameSep}{0pt}						% correct colorbox width (package: framed)

\usepackage{xcolor}
\usepackage{graphicx}
\usepackage{tikz}
\usetikzlibrary{positioning}
\usetikzlibrary{arrows}

\usepackage{enumitem}
\setenumerate[1]{label=\roman*)}			% roman numbered lists
\setitemize			{leftmargin=*}
\setenumerate		{leftmargin=*}


\usepackage{chapterthumb}							% use chapter thumbs
\setkomafont{chapterthumb}						% use small font for chapter thumbs
	{\normalfont\sffamily}
\renewcommand{\chapterthumbheight}		% make chapter thumbs thinner
	{1em}
	
\setkomafont{chapterentry}						% no sans-serif but bold chapter entries in toc
	{\normalfont\bfseries}
\setkomafont{chapterentrypagenumber}	% no bold chapter page entry numbers in toc
	{\normalfont}
	
\renewcommand{\footnoterule}{					% increase spacing below footnote line
	\noindent\rule{0.4\textwidth}{.4pt}
	\vspace{0.6em}}
\deffootnote{1em}{1em}								% footnote in normal textsize
	{\thefootnotemark\hspace{0.5em}}
	
\usepackage{scrpage2}
\setkomafont{pageheadfoot}
	{\normalfont\itshape}
\pagestyle{scrheadings}								% switch to modified style

\renewcommand{\chaptermark}[1]
	{\markboth{\thechapter~#1~}{}}
\renewcommand{\sectionmark}[1]
	{\markright{\thesection\autodot~#1}}
					
\usepackage{titlesec}									% design chapter headings
\titleformat{\chapter}[display]
	{\normalfont\sffamily\Huge\bfseries\flushright\fontsize{70}{0}\selectfont}
	{\thechapter}
	{-0pt}{\Huge}
\setkomafont{section}{\normalfont\Large\bfseries}
\setkomafont{subsection}{\normalfont\large\bfseries}
\titlespacing*{\chapter}{0pt}{0pt}{50pt}

\ifthenelse{\equal{\useRoman}{1}}									% roman chapter numbering
	{\renewcommand{\thechapter}{\Roman{chapter}}}{}
					
\clubpenalty 					= 10000					%
\widowpenalty 				= 10000					% avoid widow and club lines
\displaywidowpenalty 	= 10000					%

\makeatletter
\@beginparpenalty			=	10000					% avoid page breaks before lists
\makeatother

\definecolor{linkblue}{rgb}{0.0, 0.0, 0.3}
\definecolor{citeblue}{rgb}{0.0, 0.0, 0.5}
\usepackage[pdfstartpage				= 1,							% default opening start page 
						pdfstartview				= FitB, 					% default opening zoom
						pdftitle						= {\docTitel},		% document title
						pdfauthor						= {\docAutor}, 		% document author
						colorlinks					= true, 					% print links colored
						bookmarks						= true, 					% use bookmarks
						bookmarksopen				= true,						% open bookmarks by default
						bookmarksnumbered		= true, 					% number bookmarks
						linkcolor						= linkblue,					% color for links
						citecolor						= citeblue,				% color for cite links
						pdfdisplaydoctitle	= true						% display document title instead of file name
					 ]{hyperref}

% mathematical
\usepackage{amssymb}
\usepackage{amsmath}
\usepackage{amsthm}
\usepackage{latexsym}
\usepackage{stmaryrd}
\usepackage{esint}										% provides \ointclockwise etc.
\usepackage{mathtools}								% provides \xRightarrow[]{}, \mathclap, \shortintertext and \bigtimes

\makeatletter																			% alternate closed root symbol
\ifthenelse{\equal{\usroptsqrt}{1}}{%
\let\oldr@@t\r@@t
\def\r@@t#1#2{%
\setbox0=\hbox{$\oldr@@t#1{#2\,}$}\dimen0=\ht0
\advance\dimen0-0.2\ht0
\setbox2=\hbox{\vrule height\ht0 depth -\dimen0}%
{\box0\lower0.65pt\box2}}
\LetLtxMacro{\oldsqrt}{\sqrt}
\renewcommand*{\sqrt}[2][\ ]{\oldsqrt[#1]{#2}}}{}
\makeatother