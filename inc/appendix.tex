\chapter{Anhang}
\renewcommand*{\thetmpsatz}{\thechapter.\arabic{tmpsatz}}
\setcounter{tmpsatz}{0}
An dieser Stelle wollen wir einige wichtige S�tze aus anderen Vorlesungen festhalten, wie mehrmals ben�tigt und referenziert werden. Da diese nur der �bersicht dienen, werden wir auf Beweise dieser S�tze an dieser Stelle verzichten.

\begin{satz}[Totale Wahrscheinlichkeit]\label{appendix:totwkeit}
Sei $(\Omega, \sA, P)$ ein Wahrscheinlichkeitsraum und $(B_i)_{i \in I}$ eine h�chstens abz�hlbare Zerlegung von $\Omega$. Ferner gelte $P(B_i) > 0$ f�r alle $i \in I$. F�r alle $A \in \sA$ gilt dann
\begin{align*}
P(A) &= \sum_{i \in I} P(B_i)P(A \mid B_i)\text{.}
\end{align*}
\end{satz}

\begin{beweis}
Der Beweis findet sich in \cite[Satz I.7.3]{WT}.
\end{beweis}

\begin{satz}[Beppo Levi I / Monotone Konvergenz]\label{appendix:beppolevi1}
Es sei $(\Omega, \sA, \mu)$ ein Ma�raum und $f_n\colon \Omega \to [0, \infty]$ f�r $n \geq 1$ eine Folge messbarer Funktionen mit $f_n \nearrow f$. Dann folgt, dass $f$ messbar und nicht-negativ ist. Au�erdem gilt
\begin{align*}
\int f~\dd \mu &= \lim_{n \to \infty} \int f_n~\dd \mu = \sup_{n \geq 1} \int f_n~\dd \mu\text{.}
\end{align*}
\end{satz}

\begin{beweis}
Der Beweis findet sich in \cite[Satz I.11.13]{WT}.
\end{beweis}

%\begin{satz}[Beppo Levi II]\label{appendix:beppolevi2}
%Es sei $(\Omega, \sA, \mu)$ ein Ma�raum und $(f_n) \subset \sL^1(\mu)$ eine Folge integrierbarer Funktionen, so dass $\mu$-fast �berall $f_n \nearrow f$ f�r eine Abbildung $f\colon \Omega \to \R$ gilt. Dann folgt
%\begin{align*}
%\lim_{n \to \infty} \int f_n~\dd\mu &= \int f~\dd\mu = \int f^+~\dd\mu - \int f^-~\dd\mu\text{.}
%\end{align*}
%\end{satz}
%
%\begin{beweis}
%Der Beweis findet sich in \cite[Satz I.11.13]{WT}.
%\end{beweis}

\begin{lemma}[Fatou]\label{appendix:fatou}
Es sei $(\Omega, \sA, \mu)$ ein Ma�raum und $f_n\colon \Omega \to [0, \infty]$ f�r $n \geq 1$ eine messbare Funktionenfolge. Dann gilt
\begin{align*}
\int \liminf_{n \to \infty} f_n~\dd\mu &\leq \liminf_{n \to \infty} \int f_n~\dd \mu\text{.}
\end{align*}
\end{lemma}

\begin{beweis}
Der Beweis findet sich in \cite[Lemma I.11.14]{WT}.
\end{beweis}

\begin{satz}[Lebesgue / Dominierte Konvergenz]\label{appendix:lebesgue}
Es sei $(\Omega, \sA, \mu)$ ein Ma�raum, $f_n\colon \Omega \to \R \cup \{\pm\infty\}$ f�r $n \geq 1$ eine Folge messbarer Funktionen und $f, g\colon \Omega \to \R \cup \{\pm\infty\}$ messbare Abbildungen mit $f_n \to f$ und $|f_n| \leq g$ f�r alle $n \geq 1$. Ist $g$ bez�glich $\mu$ integrierbar, so gilt dies auch f�r $f$ und es folgt
\begin{align*}
\int \lim_{n \to \infty} f_n~\dd\mu &= \int f~\dd\mu = \lim_{n \to \infty} \int f_n~\dd\mu\text{.}
\end{align*}
\end{satz}

\begin{beweis}
Der Beweis findet sich in \cite[Lemma I.11.16]{WT}.
\end{beweis}

\begin{lemma}[Borel-Cantelli II]\label{appendix:borelcantelli2}
Es sei $(\Omega, \sA, P)$ ein Wahrscheinlichkeitsraum und $(A_n)_{n \geq 1} \subset \sA$ unabh�ngig. Dann gilt
\begin{align*}
\sum_{n=1}^\infty P(A_n) = \infty &\Longrightarrow P\left(\limsup_{n \to \infty} A_n\right) = 1\text{.}
\end{align*}
F�r unabh�ngige Folgen $(A_n)_{n \geq 1} \subset \sA$ gilt damit insbesondere $P(\limsup A_n) \in \{0,1\}$ und $P(\liminf A_n) \in \{0,1\}$.
\end{lemma}

\begin{beweis}
Der Beweis findet sich in \cite[Lemma II.6.1]{WT}.
\end{beweis}